\section{Review of Complex Analysis}
The examples of minimal surfaces given previously pretty much cover the early developments in minimal surface theory. However the sheer lack of them was proving to be a bottleneck to their study. What was needed was a new way of attacking them and generating new ones. The answer came in the form of Complex Analysis.

In order to proceed we will very briefly give an overview of the main aspects of complex analysis we will be using. Further material on this subject is available in \cite{BCA}.

\subsection{The Cauchy-Riemann Equations}
These lie at the heart of complex analysis and allow us to differentiate complex functions.

The differential of a complex function is calculated using the same limit as for a real function:
\begin{displaymath}
\lim_{z\rightarrow z_0}\frac{f(z_0)-f(z)}{z-z_0}
\end{displaymath}
However we can let $z$ approach $z_0$ in any way we please. If
\begin{displaymath}
z_o = x_0 + i y_0
\end{displaymath}
we can then let $z = x_0 + h + i y_0$ or $z = x_0 + i (y_0+k)$ with $h$ and $k$ real.

However $f'(z)$ must be uniquely defined no matter how we let $z$ approach $z_0$ and so evaluating the limits of the two above cases we get

\begin{displaymath}
f'(z_0) = \frac{\partial u}{\partial x} + i \frac{\partial v}{\partial x} = \frac{\partial v}{\partial y} - i \frac{\partial u}{\partial y}
\label{CR}
\end{displaymath}
Comparing real and imaginary parts, we find that

\begin{displaymath}
\frac{\partial u}{\partial x} = \frac{\partial v}{\partial y} \ \ \  \frac{\partial v}{\partial x} =  -\frac{\partial u}{\partial y}
\end{displaymath}

These are the Cauchy-Riemann equations which hold for differentiable complex functions.

\subsection{Harmonic Functions}
Harmonic functions uncover a beautiful aspect of minimal surfaces, showing that they come in families.

Given a real valued function $\phi(x,y)$ we say that $\phi$ is a harmonic function if it satisfies \emph{Laplaces Equation}

\begin{displaymath}
\frac{\partial^2 \phi}{\partial x^2} + \frac{\partial^2 \phi}{\partial y^2} = 0 
\end{displaymath}

If $f$ is an analytic function (complex differentiable) defined in a domain $D$ and $f(z) = u(x,y)+iv(x,y)$, then we can show both u and v are harmonic in D. 

We know

\begin{displaymath}
f'(z) = \frac{\partial u}{\partial x} + i \frac{\partial v}{\partial x} = \frac{\partial v}{\partial y} - i \frac{\partial u}{\partial y}
\end{displaymath}

Setting $f' = U +iV$ and using the Cauchy-Riemann equations for U, V the partial derivatives of U and V exist and satisfy:

\begin{equation}
\frac{\partial U}{\partial x} = \frac{\partial V}{\partial y}
\label{CR_A}
\end{equation}
\begin{equation}
\frac{\partial V}{\partial x} =  -\frac{\partial U}{\partial y}
\label{CR_B}
\end{equation}

Since

\begin{displaymath}
U = \frac{\partial u}{\partial x} = \frac{\partial v}{\partial y}\ \ \  V = \frac{\partial v}{\partial x} = -\frac{\partial u}{\partial y}
\end{displaymath}

Subbing into \ref{CR_A} gives 

\begin{displaymath}
\frac{\partial}{\partial x}(\frac{\partial u}{\partial x}) 
= \frac{\partial}{\partial x}(\frac{\partial v}{\partial y})
= \frac{\partial}{\partial y}(\frac{\partial v}{\partial x})
= -\frac{\partial}{\partial y}(\frac{\partial u}{\partial y})
\end{displaymath}

So 

\begin{displaymath}
\frac{\partial^2 u}{\partial x^2} + \frac{\partial^2 u}{\partial y^2} = 0
\end{displaymath}

Subbing into \ref{CR_B} gives 

\begin{displaymath}
\frac{\partial}{\partial x}(\frac{\partial v}{\partial x}) 
= -\frac{\partial}{\partial x}(\frac{\partial u}{\partial y})
= -\frac{\partial}{\partial y}(\frac{\partial u}{\partial x})
= -\frac{\partial}{\partial y}(\frac{\partial v}{\partial y})
\end{displaymath}

So 

\begin{displaymath}
\frac{\partial^2 v}{\partial x^2} + \frac{\partial^2 v}{\partial y^2} = 0
\end{displaymath}

So if $f$ is an analytic function then its real and imaginary parts are harmonic functions.