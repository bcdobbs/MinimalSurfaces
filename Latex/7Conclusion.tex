\section{Conclusion}
The first half of this project was concerned with minimal surfaces in their classical $\mathbb R^3$ setting. We saw minimal surfaces in their real world soap film origins before introducing Differential Geometry, and noting that all Minimal Surfaces have zero mean curvature.

Having this property that characterised all minimal surfaces allowed us to set up a partial differential equation known as the Minimal Surface Equation, at which point we imposed various algebraic and geometric constraints in order to be able to solve the equation and find Minimal Surfaces. Although this approach gave some nice results; notably that the Helicoid is the only ruled minimal surface and the Catenoid is the only minimal surface of rotation, it failed to provide large numbers of different surfaces since it hinges on being able to solve the related partial differential equations which often proves too difficult.

 In order to continue our exploration of these surfaces we therefore looked to complex analysis. The power of complex analysis opened up the underlying beauty of minimal surfaces - their harmonic conjugate families, isothermal parameters that simplify the equations, deep links between minimal surfaces and the Gauss Map and of course the jewel that is the \emph{Weierstrass-Enneper Representation} which allows us to produce new minimal surfaces virtually at will.

Having laid down this classical base for minimal surfaces we then left the confines of $\mathbb R^3$ and looked at an example of how the $\mathbb R^3$ result, showing that the only minimal ruled surface is the Helicoid, generalises to hypersurfaces in higher dimensions. This required a large amount of Riemannian Manifold geometry which has proved fascinating.

Whenever possible, illustrations of the surfaces considered have been included. These were produced in Maple and whilst obviously nice to look at, it is important to realise that this relatively recent computer aided tool for rendering surfaces has been responsible for the rapid advances in the field in recent decades. Such images do not provide solid mathematical proof however they can provide the impetus to make statements which the mathematician can then attempt to prove. A prime example of this was the discovery in 2004 of an \emph{Embedded Genus One Helicoid} (a Helicoid with a handle). Computer visualisation was used to suggest the existence of such an object. However a computer has no way of knowing whether somewhere out towards infinity the surface would intersect itself. Therefore it was left to mathematicians to prove that this surface was in fact embedded. See \cite{HOFF}

Of course there is a huge wealth of material on minimal surfaces that we have not touched upon. It would have been nice to provide an overview of the calculus of variations to approach minimal surfaces and show that they do in fact form at turning points of surface area. A treatment of this can be found in \cite{OPR}. Also minimal surfaces are part of a much larger group of surfaces known as \emph{Constant Mean Curvature Surfaces}. As the name suggests these are surfaces that have mean curvature of a fixed value. The list of interesting topics in this area is virtually endless.
